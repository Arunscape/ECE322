For task two in this lab, the BoilerShutdown application was tested using
a simplified cause-effect graph from the one which was given, from which a
decision table was made, and test cases were created.
BoilerShutdown is a GUI application written in Java, which has 10 checkbox
inputs labelled alphabetically from A to J. These are defined as:
\begin{itemize}
	\item A - Water level meter failure during operation
	\item B - Steam rate meter failure during operation
	\item C - Water level out of range
	\item D - Instrumentation system failure
	\item E - Communication link failure
	\item F - Steam rate meter failure during startup
	\item G - Water level meter failure during startup
	\item H - Multiple pumps failure (more than one)
	\item I - Confirmed keystroke entry
	\item J - Confirmed ``shut now'' message
\end{itemize}

The given cause-effect graph for Task 2 was simplified, and can be found in
Appendix B. By tracing back the graph, we were able to determine that inputs
A, B, and C had no effect on whether the Boiler would shut down or not.
Furthermore, We were able to eliminate almost all intermediate nodes, since it
was found that the boiler effectively shuts down if any of the inputs D, E, F,
G, H, I, or J $=1$ and it does not shut down if all of the inputs from D-J
$=0$. From the new cause-effect graph, the following rules were derived:
\begin{itemize}
	\item IF D OR E OR F OR G OR H OR I OR J THEN Boiler Shutdown
	\item IF NOT D AND NOT E AND NOT F AND Not G AND NOT H AND NOT I AND
	      NOT J THEN Boiler not Shutdown
\end{itemize}

From the rules defined above, a decision table was made:
\vspace{20pt}

\begin{adjustbox}{center}
	\begin{tabular}{lllllllll}
		                    & 1 & 2 & 3 & 4 & 5 & 6 & 7 & 8 \\
		Causes              &   &   &   &   &   &   &   &   \\
		\cline{1-1}
		A                   & x & x & x & x & x & x & x & x \\
		B                   & x & x & x & x & x & x & x & x \\
		C                   & x & x & x & x & x & x & x & x \\
		D                   & 1 & x & x & x & x & x & x & 0 \\
		E                   & x & 1 & x & x & x & x & x & 0 \\
		F                   & x & x & 1 & x & x & x & x & 0 \\
		G                   & x & x & x & 1 & x & x & x & 0 \\
		H                   & x & x & x & x & 1 & x & x & 0 \\
		I                   & x & x & x & x & x & 1 & x & 0 \\
		J                   & x & x & x & x & x & x & 1 & 0 \\
		                    &   &   &   &   &   &   &   &   \\
		Effects             &   &   &   &   &   &   &   &   \\
		\cline{1-1}
		Boiler shutdown     & 1 & 1 & 1 & 1 & 1 & 1 & 1 & 0 \\
		Boiler not shutdown & 0 & 0 & 0 & 0 & 0 & 0 & 0 & 1
	\end{tabular}%
\end{adjustbox}


\vspace{20pt}
From the decision table above, the following test cases were made:
\vspace{20pt}

\begin{adjustbox}{center}
	\begin{tabular}{lllllllllllll}
		Testid & A & B & C & D & E & F & G & H & I & J & Expected &
		Actual                                                           \\ \hline
		1      & 0 & 0 & 0 & 1 & 0 & 0 & 0 & 0 & 0 & 0 & FAIL     & FAIL \\
		2      & 0 & 0 & 1 & 0 & 1 & 0 & 0 & 0 & 0 & 0 & FAIL     & FAIL \\
		3      & 0 & 1 & 0 & 0 & 0 & 1 & 0 & 0 & 0 & 0 & FAIL     & FAIL \\
		4      & 0 & 1 & 1 & 0 & 0 & 0 & 1 & 0 & 0 & 0 & FAIL     & FAIL \\
		5      & 1 & 0 & 0 & 0 & 0 & 0 & 0 & 1 & 0 & 0 & FAIL     & FAIL \\
		6      & 1 & 0 & 1 & 0 & 0 & 0 & 0 & 0 & 1 & 0 & FAIL     & FAIL \\
		7      & 1 & 1 & 0 & 0 & 0 & 0 & 0 & 0 & 0 & 1 & FAIL     & FAIL \\
		8      & 1 & 1 & 1 & 0 & 0 & 0 & 0 & 0 & 0 & 0 & Ok       & Ok
	\end{tabular}%
\end{adjustbox}
\vspace{20pt}

Normally, failed test cases would be highlighted in red. However, all test
cases passed in this scenario.


\subsection{Discussion}
While no failures in the software were caught, this is not neccessarily an
indication of poor tests. It appears that there is good coverage, since
there is one test case for each input that should cause the boiler to shut down,and an additional test case where the boiler should not shut down.
In terms of number of test cases, this is definitely more
efficient compared to previous methods learned like EPC, weak n x 1, and dirty
testing. Only 8 test cases were made to test the functionality, while other
methods would have resulted in far more test cases. Deriving test cases from
the decision table for this piece of software seemed
more appropriate than for the InsuranceCalculator program, since the program
constrains the user to only enter valid inputs. This avoids scenarios like in
Task 1 where the the test cases had a bunch of valid inputs, but missed the
cases where an input was invalid. Compared to other black-box testing methods,
the cause-effect diagram and decision tables seems most effective for testing
this program, since there are not really any boundaries to check, and the
inputs are always valid. In this case, we're only concerned about the inputs
(causes) and their direct effect, and there are not really any edge cases to
consider. Combinatorial testing, and testing all possible combinations are two
other testing techniques that my be considered for testing this program. 

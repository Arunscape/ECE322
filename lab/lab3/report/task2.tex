For task two in this lab, the RemoteCar application was tested using both the
EPC method and the weak n x 1 strategy. RemoteCar is a command-line program
written in Java, which takes two inputs, $x$, and $y$, which represent a
point on the Cartesian plane.
If $(x,y)$ falls on the circle with radius $r=1$, centered about the origin,
the program will output`Ok.', or `Out of range!'. The input arguments must
be real numbers, or the program should output an error letting the user know
that the argument entered was not a number.

\subsection{Discussion}
The domain approximation is somewhat effective. It seems to work best if
test inputs chosen are far away from the boundaries, since there is a risk
of a false negative test result if the chosen test input is too close to the
boundary. This can happen when the test case chosen is outside of the
approximated boundary, while it is still in the actual subdomain's boundary.
In such a scenario, we might expect a test case to fail using the
approximation, yet if the program is running correctly, the test case will
pass. The configuration could have been adjusted to yield higher accuracy
of the subdomain. For example, we could have increased the number of linear
boundaries. Using the EPC strategy, the complexity of testing is unaltered.
There are still 17 test cases, and they would all be the same. However,
using the weak n x 1 strategy, the complexity would be increased.
We would have to make $b \times (2 + 1) + 1$ test cases, where b is
the number of linear boundaries we choose to use for the approximation.
In other words, we would need an additional 3 test cases for each boundary
we add to increase the accuracy.

For this problem, it seems that the EPC testing strategy is more accurate.
Using the weak n x 1 strategy, we ran into some cases where the test input
was outside of our approximated boundary, yet it was inside the actual
subdomain. This resulted in 4 test case failures, which were not actual
errors with the program itself, but rather due to the innacurate linear
approximation of the subdomain. As a result, no actual errors were identified
in the application. In this case, with $m=4$ boundaries, the number of test
cases for the EPC and weak n x 1 strategies were similar. However,
the weak n x 1 strategy could have a lot more test cases if we added more
linear boundaries to make a better approximation of the problem's subdomain.

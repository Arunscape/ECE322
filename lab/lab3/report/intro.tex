The purpose of this lab was to serve as a practical
introduction to some more black box testing techniques. In this lab, the
testing methods introduced were the Extreme Point Combination (EPC), and the
Weak n x 1 strategy methods. We tested two programs written in Java using both
of these testing strategies. The first application, named Drone takes in three
arguments, and outputs either `Success!', `Failure', or an error message based
on whether the arguments are integers $\geq 0$, and their sum is less than
$k=100$. The second program, named RemoteCar takes in two arguments which
represent points on a Cartesian plane, and outputs either `Ok', `Out of range',
or an error message based on whether the point is on a circle of radius 1 about
the origin. The idea for EPC testing is to identify the input domain limits,
and produce all possible combinations of inputs with each of the input
variables taking on a minimum value, slightly under minimum, a maximum value,
and slightly over maximum value. One additional test case is added somewhere
within the valid subdomain to generate a total of $4^n + 1$ test cases, where
n is the dimension of inputs, or put more simply, the number of input
variables. With the weak n x 1 strategy, we attempt to find linear boundaries 
of the problem using domain analysis. $n$ points are selected on each linear
boundary, where $n$ is the number of input variables, and one additional point 
is chosen outside of each boundary. An additional point within the boundary is
chosen if the boundary is open, and if the boundary is instead closed an 
additional point outside the boundary is chosen. In the end, with the weak
n x 1 strategy, $b \times (n + 1)$ test cases are generated, where b is the 
number of linear boundaries, and n is the dimensionality, or the number of
inputs.

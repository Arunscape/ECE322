In this lab, we were introduced to two more black box testting strategies. The
techniques learned were the Extreme Point Combination, or EPC strategy, and the
weak n x 1 strategy. It would not be fair to say that one testing strategy
is universally better than the other, since it would be better to match
each problem at hand to the appropriate testing strategy. The weak n x 1
strategy seems to shine when the problem's subdomain is already linear in
nature. This way, no approximation needs to be made, and we get very few
test cases $b \times (n + 1) + 1$, where b is the number of boundaries, and
n is the number of input variables. The weak n x 1 strategy appears to not
be very effective however, when the problem's subdomain is approximated as
a collection of linear boundaries. This configuration appears to result in
false negative test cases, and additionally, if more input boundaries are
used to increase the accuracy of the input domain, it is still not perfect,
and furthermore, $n + 1$ additional test cases need to be added for each
additional boundary, which can result in more test cases required to be carried
out, which all still have the potential issue of resulting in a false
negative result. On the other hand, the EPC strategy might seem tedious
at first, generating $4^n + 1$ test cases, which might seem excessive.
However, it seems to overall be more accurate than the weak n x 1 stratrgy.
There were no false positive test cases, and furthermore, more failure
scenarios were found, even though they are likely from the same root cause.
(For the Drone program, we found using the EPC strategy 2 types of failures,
where the program outputs both `Success!'
or `Failure!' when $x_2$ was negative instead of an error message, however,
with the weak n x 1 strategy we only saw one case where
`Success!' was outputted mistakenly. From a purely subjective standpoint,
I personally prefer the EPC testing strategy over weak n x 1, since
even though a lot of test cases are typically generated, this step can be
automated using a script, and not lot of thought, since only four values
(min, slightly under min, max, slightly over max) plus one additional
valid test case needs to be thought of. Computers are fast, so we can simply
generate most of the test cases without much thought, and then compare the
results with ease using more automation. However, using the weak n x 1
strategy, one has to really think about each boundary, and choose $n$ points on
it, plus one point slightly outside each boundary, and additionally, one final
test case. Additionally, weak n x 1 testing does not seem to work well for
subdomains that are non-linear in nature, while the EPC strategy seems to
overall work well and reliably.

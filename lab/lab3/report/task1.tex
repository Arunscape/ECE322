For task one in this lab, the InsuranceCalculator application was tested using
a cause-effect graph from which a decision table and test cases were made.
InsuranceCalculator is a GUI program
written in java, which takes in three inputs:
\begin{itemize}
	\item gender: Male/Female
	\item age: integer
	\item: claims: integer
\end{itemize}
Depending on the value of the inputs, the program will output either \$750,
\$1000, \$1500, \$3000, or ERROR.

From the following rules, a cause-effect graph was generated, and can be found
in Appendix A.
\begin{itemize}
	\item IF sex = Male AND age $<25$ AND claims $=0$ THEN premium $=\$1500$
	\item IF sex = Male AND $25\leq$ age $<65$ AND claims $=0$ THEN premium
	      $=\$1000$

	\item IF sex = Female AND age $<65$ AND claims $=0$ THEN premium
	      $=\$750$
	\item IF age $\geq 65$ AND claims $=0$ THEN premium $=\$1500$
	\item IF claims $\geq 1$ THEN premium $=\$3000$
\end{itemize}

From the cause-effect graph found in Appendix A, the following decision table
was made:
\vspace{20pt}

\begin{adjustbox}{center}
	\begin{tabular}{llllll}
		        & 1 & 2 & 3 & 4 & 5 \\ \cline{2-6}
		Causes  &   &   &   &   &   \\ \cline{1-1}
		A       & 0 & 1 & 1 & x & x \\
		B       & 1 & 0 & 0 & x & x \\
		C       & x & 0 & 1 & 0 & x \\
		D       & x & 1 & 0 & 0 & x \\
		E       & 0 & 0 & 0 & 1 & x \\
		F       & 1 & 1 & 1 & 1 & 0 \\
		G       & 0 & 0 & 0 & 0 & 1 \\
		        &   &   &   &   &   \\
		Effects &   &   &   &   &   \\ \cline{1-1}
		Y1      & 1 & x & x & x & x \\
		Y2      & x & 1 & x & x & x \\
		Y3      & x & x & 1 & 1 & x \\
		Y4      & x & x & x & x & 1
	\end{tabular}%
\end{adjustbox}

From the decision table above, the following test cases were made:
\vspace{20pt}

\begin{adjustbox}{center}
	\begin{tabular}{cccccc}
		Test ID & Sex & Age & Claims & Expected & Actual \\ \hline
		1       & F   & 64  & 0      & \$750    & \$750  \\
		2       & M   & 25  & 0      & \$1000   & \$1000 \\
		3       & M   & 24  & 0      & \$1500   & \$1500 \\
		4       & F   & 66  & 0      & \$1500   & \$1500 \\
		5       & F   & 54  & 1      & \$3000   & \$3000 \\
	\end{tabular}%
\end{adjustbox}
Normally, failed test cases would be highlisghted in red. However, none of the
test cases failed in this case.

\subsection{Discussion}
Using both the EPC and the weak n x 1 strategy, we see that for all of the
failed test cases, the input variable $x_2$ is a negative. For negative
inputs, the program is expected to give an error telling the user that
negative inputs are not valid. Instead, the program outputs `Success!'
or `Failure!' when $x_2 < 0$. Using the EPC strategy, we see
2 types of failures, where the program outputs both `Success!'
or `Failure!'; however, with the weak n x 1 strategy we only saw the case where
`Success!' is outputted mistakenly. It should be noted however, that both
testing strategies found an error which seems to stem from the same issue,
(the application not checking if $x_2$ is negative, and the weak n x 1 strategy
did so with less test cases (17 versus 65). As a result, for this problem,
the weak n x 1 strategy seemed to be both more efficient and effective,
since the same problem was caught using the EPC strategy but with far fewer
test cases. \todo{kms}

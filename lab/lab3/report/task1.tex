For task one in this lab, the InsuranceCalculator application was tested using
a cause-effect graph from which a decision table and test cases were made.
InsuranceCalculator is a GUI program
written in java, which takes in three inputs:
\begin{itemize}
	\item gender: Male/Female
	\item age: integer
	\item: claims: integer
\end{itemize}
Depending on the value of the inputs, the program will output either \$750,
\$1000, \$1500, \$3000, or ERROR.

From the following rules, a cause-effect graph was generated, and can be found
in Appendix A.
\begin{itemize}
	\item IF sex = Male AND age $<25$ AND claims $=0$ THEN premium $=\$1500$
	\item IF sex = Male AND $25\leq$ age $<65$ AND claims $=0$ THEN premium
	      $=\$1000$

	\item IF sex = Female AND age $<65$ AND claims $=0$ THEN premium
	      $=\$750$
	\item IF age $\geq 65$ AND claims $=0$ THEN premium $=\$1500$
	\item IF claims $\geq 1$ THEN premium $=\$3000$
\end{itemize}

From the cause-effect graph found in Appendix A, the following decision table
was made:
\vspace{20pt}

\begin{adjustbox}{center}
	\begin{tabular}{llllll}
		        & 1 & 2 & 3 & 4 & 5 \\ %\cline{2-6}
		Causes  &   &   &   &   &   \\ \cline{1-1}
		A       & 0 & 1 & 1 & x & x \\
		B       & 1 & 0 & 0 & x & x \\
		C       & x & 0 & 1 & 0 & x \\
		D       & x & 1 & 0 & 0 & x \\
		E       & 0 & 0 & 0 & 1 & x \\
		F       & 1 & 1 & 1 & 1 & 0 \\
		G       & 0 & 0 & 0 & 0 & 1 \\
		        &   &   &   &   &   \\
		Effects &   &   &   &   &   \\ \cline{1-1}
		Y1      & 1 & x & x & x & x \\
		Y2      & x & 1 & x & x & x \\
		Y3      & x & x & 1 & 1 & x \\
		Y4      & x & x & x & x & 1
	\end{tabular}%
\end{adjustbox}

From the decision table above, the following test cases were made:
\vspace{20pt}

\begin{adjustbox}{center}
	\begin{tabular}{cccccc}
		Test ID & Sex & Age & Claims & Expected & Actual \\ \hline
		1       & F   & 64  & 0      & \$750    & \$750  \\
		2       & M   & 25  & 0      & \$1000   & \$1000 \\
		3       & M   & 24  & 0      & \$1500   & \$1500 \\
		4       & F   & 66  & 0      & \$1500   & \$1500 \\
		5       & F   & 54  & 1      & \$3000   & \$3000 \\
	\end{tabular}%
\end{adjustbox}
Normally, failed test cases would be highlighted in red. However, none of the
test cases failed in this case.

\subsection{Discussion}
While no failures in the software were caught, this is not neccessarily an
indication of poor tests. It does appear that there is a good coverage, since
there is one test case which tests the expected output for each rule defined in
the software specification, and it does so with relatively few test cases.
(Only 5 in this case). However, using other black-box testing methods learned
earlier like dirty testing, EPC, or weak n x 1 strategies, some more scenarios
are covered, which were not covered using the cause-effect and decision table
tools. For example, none of the test cases tested negative valued
inputs for the age or total claims fields, which allow negative inputs. In this
case, the application behaves as expected, outputting an ERROR message,
however, this scenario was not covered when using the techniques learned in
this lab. Relative to the other black-box testing techniques learned, there are
fewer test cases, which makes efficient test cases from cause-effect graphs and
decision tables. However, from a subjective standpoint, using this method
demanded more effort for creating test cases versus other techniques
previously learned. The black box testing techniques previously learned also
seem to cover more scenarios that were missed using this technique.

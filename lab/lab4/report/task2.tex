For the second part of the lab, we are to assume that we have a system with
three independent variables: A, B, C. Each variable has three possible values:
0, 1, 2. There is no actual testing for this portion, it is a conceptual
exercise.

In total, there are $3 \times 3 \times 3  = 27$ test cases if we were to do
combinatorial testing. We are looking for a standard orthogonal array that
can fit $3^3$. From \url{http://neilsloane.com/oadir/}, we find that
the $L_9(3^4)$ standard orthogonal array works. 
Below is the set of test cases using the orthogonal array mentioned above:

\inputminted{text}{resources/standard3^4}

The PICT program (\url{https://github.com/microsoft/pict}) was used to generate
test cases with the following input:

\inputminted{text}{resources/pict}

It should be noted that strangely, there seems to be different outputs
depending on the operating system that pict is run on.

On GNU/Linux systems, the output is as follows:
\inputminted{text}{resources/pictlinux}

On Windows, the output is as follows:
\inputminted{text}{resources/pictwindoze}

On macOS, the output is as follows:
\inputminted{text}{resources/pictmacos}

What is strange is that on the Windows and macOS systems, 10 test cases are
generated, while on GNU/Linux, 11 test cases are generated. 

Given the inconsistent outputs of the pict tool depending on the operating
system that the program is run on, I am not sure about the effectiveness of the
tool for test case generation. Ideally, the test cases generated should be
consistent, since this they are not supposed to be chosen randomly. If the
program worked consistently, however, it would be quite effective since not
much effort would be required to come up with the test cases, and we have the
guarantee that every pair of inputs is tested. 
Compared to the orthogonal array, the pict tool generated more test cases,
which is another consideration to take into account. 


Pairwise testing is fairly useful, especially when you consider it versus
combinatorial testing. In this toy example, the number of test cases was not
reduced that much, (27 for combinatorial testing, versus 9 with an orthogonal
table), but one can imagine the reduction in test cases when there are more
inputs, and more possible values that these inputs can take on. Compared with
random combinations, pairwise testing gives us the guarantee that we are
testing every single pair of input factors, whereas random combinations
does not have this guarantee, and there are far more combinations to choose
from. However, random combinations can by chance reveal errors that pairwise
testing would not catch.

Pairwise testing catches errors where two inputs interact with each other.
It does not take into account the effect that multiple inputs might have 
together. For example, imagine an application that requires three inputs to
be in a certain state before something gets activated. Using pairwise testing
there is a possibility that we would miss testing the part of the app,
where three inputs are in the state for something to happen, since
the pairwise tests would only focus on the combination of pairs of the inputs.
In this way, core functionality of the application can be missed in testing.
Furthermore, with pairwise testing it is possible to miss the more 
probable combinations when selecting the test data. However, given the reduction
in the number of test cases versus exhaustive testing, not to mention the
cost of running so many tests, these weaknesses may be justifiable to some
testers, given that pairwise testing usually results yields higher defect
detection rates, increases the test coverage, all while using significantly
less test cases. As mentioned above, to cover all possible inputs for the
application in this example, a total of 27 test cases are required. 


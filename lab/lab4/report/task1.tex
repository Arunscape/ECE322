For task one in this lab, the Bisect application was tested using White box
testing methods. Bisect is a program written in Java, which
uses the well-known bisection algorithm in mathematics to attempt to find the
root of any polynomial in a given interval that crosses $x=0$ somewhere. That
is, wherever the polynomial intersects $x=0$.
The algorithm which the program implements is outlined as follows:

\begin{enumerate}
	\item Calculate c, the midpoint of the interval: $ c = \frac{a+b}{2} $
	\item Calculate the polynomial's value at $f(c)$
	\item If $|f(c)|$ is within the tolerance, stop
	\item Check the sign of $f(c)$ and replace either $(a, f(a))$ or $(b, f(b))$
	      with $(c, f(c))$ such that the interval crosses $x=0$ somewhere and repeat
	      until step 3 quits
\end{enumerate}

Since this is white box testing, we have access to the source code for this
application. By inspecting the source code, the following control flow graph was
generated for the application. The numbers in each node represent the line
numbers in the source code in \textbf{Bisect.java}.

\begin{adjustbox}{center, height=\textheight}
	\digraph{controlflow}{
		rankdir=TB;
		idk [label = "56-68"];
		"52-54"->idk;
		70->71->idk;
		idk->60;
		60->61;
		60->"64-65"->66;
		66->67->70;
		66->69->70;
		71->73->74;
		73->77;
	}
\end{adjustbox}

By inspecting the source code and understanding the bisection algorithm,
some JUnit test cases were generated. These test cases can be found in Appendix
\ref{junit}. A summary of these test cases is outlined in the table below.
\vspace{1cm}

\begin{adjustbox}{center}
	\begin{tabularx}{0.9\paperwidth}{c X c c c c c l l}
		Test Id & Description                                                                              & tolerance & max iterations & polynomial & $x_1$ & $x_2$      & Expected  & Actual    \\ \hline
		1       & Exception is thrown when both $f(x_1)$ and $f(x_2)$ are $> 0$                            & 0.000001  & 50             & $x^2 - 1$  & -5    & 5          & Exception & Exception \\\hline
		2       & Exception is thrown when maximum iterations are exceeded                                 & 0.000001  & 1              & $x+100$    & -150  & 1000000000 & Exception & Exception \\\hline
		3       & Test constructor with tolerance                                                          & 50.0      & 50             & $x$        & -10   & 10         & 0         & 0         \\\hline
		4       & Test constructor with tolerance and max iterations                                       & 0         & 50             & $x$        & -10   & 10         & 0         & 0         \\\hline
		5       & Max iterations getter                                                                    & 0.000001  & 500            & $x$        & n/a   & n/a        & 500       & 500       \\\hline
		6       & Max iterations setter                                                                    & 0.000001  & 500            & $x$        & n/a   & n/a        & 500       & 500       \\\hline
		7       & Tolerance getter                                                                         & 0.0       & 50             & $x$        & n/a   & n/a        & 0.0       & 0.0       \\\hline
		8       & Tolerance setter                                                                         & 0.0       & 50             & $x$        & n/a   & n/a        & 0.0       & 0.0       \\\hline
		9       & Normal test case, make sure that $f(result) \leq result$, and it iterates more than once & 0.000001  & 50             & $x^2 -1$   & -1.5  & 0.5        & -1        & -1
	\end{tabularx}%
\end{adjustbox}
\vspace{1cm}

While code coverage is a useful metric for determining the quality of tests, it
should not be the only factor in determining the quality of the tests. After
all, it is possible to have all tests cover all lines and branches, but the
tests may not be checking for the correctness in execution of these statements
and branching conditions.  What is useful however, is knowing which lines of
the code have been executed in the tests. That way, if a line is missed for
example, code coverage criterion will bring to light that fact that the line of
code was not covered. Furthermore, branch coverage will also ensure that all
branches have been executed in the tests. In short, coverage criterion is
effective for letting us know when a part of the code \textit{for sure} is not
being tested. It does not however, guarantee that the correctness of the
covered statements and branches.

Based solely on the coverage criteria, it would be in my opinion foolish,
or at best naive to be completely confident in the application being bug-free.
Having well-thought-out tests that adequately cover the functionality of the
programs \textit{in addition to} the code coverage criterion, however, does
increase my confidence in the program working. Tests that are written such that 
they only cover the code coverage criterion will not adequately cover the 
correctly. The tests have to be made with the knowledge of how the Bisection
algorithm works, and testing that the program computes the result as expected.
With the test cases made in this lab, (available in Appendix \ref{junit}),
I am fairly confident that the portion of the program that does the Bisection
algorithm works correctly, since they were crafted with the knowledge
of how the Bisection algorithm works, and therefore test that the functionality
of the algorithm.

In general, the number of paths available to test is a number between
$n+1 \leq x \leq 2^n$, where n is the number of branches. This depends
on when the branches merge. Additionally, for this application,
there is a branch decision on every loop iteration. So, the number of test
cases depends on the number of maximum iterations. Even with the default value
of 50 iterations, we can see how quickly the number of tests needed for path
coverage can blow up exponentially to ridiculous proportions, which is why
path coverage is not realistic. 

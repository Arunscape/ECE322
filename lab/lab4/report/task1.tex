For task one in this lab, the Bisect application was tested using White box
testing methods. Bisect is a program written in Java, which
uses the well-known bisection algorithm in mathematics to attempt to find the
root of any polynomial in a given interval that crosses $x=0$ somewhere. That
is, wherever the polynomial intersects $x=0$.
The algotithm which the program implements is outlined as follows:

\begin{enumerate}
  \item Calculate c, the midpoint of the interval: $ c = \frac{a+b}{2} $
  \item Calculate the polynomial's value at $f(c)$
  \item If $|f(c)|$ is within the tolerance, stop
  \item Check the sign of $f(c)$ and replace either $(a, f(a))$ or $(b, f(b))$
    with $(c, f(c))$ such that the interval crosses $x=0$ somewhere and repeat
    until step 3 quits
\end{enumerate}

Since this is white box testing, we have access to the source code for this
application. By inspecting the source code, the following control flow graph was
generated for the application. The numbers in each node represent the line
numbers in the source code in \textbf{Bisect.java}

\begin{adjustbox}{center, height=\textheight}
\digraph{controlflow}{
    rankdir=TB;
    idk [label = "56-68"];
    "52-54"->idk;
    70->71->idk;
    idk->60;
    60->61;
    60->"64-65"->66;
    66->67->70;
    66->69->70;
    71->73->74;
    73->77;
}
\end{adjustbox}


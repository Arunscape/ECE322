The purpose of this lab was to serve as an introduction to white-box testing.
The goal was to become familiar with the JUnit testing library in Java, as well
as using an orthogonal table to come up with test cases for pairwise testing,
and using a tool named PICT (Pairwise Independent Combinatorial Tool) to
generate pairwise tests. White-box testing, in contrast to Black-Box testing
focuses on testing the internal parts of an application. That is, instead of
having only an outsider view of the application, we use the insight we know
about the application from knowing its source code and implementation details
to come up with tests to test the program for failures. We also look at
pairwise testing, which is an efficient way to come up with a set of test cases
that covers all possible combinations of pairs of inputs.  In the first part, a
program written in Java was tested, named Bisect.  This program implements the
well-known bisection algorithm in mathematics to find the root of a polynomial
in an interval where the polynomial crosses $x=0$. The source code was
analyzed, and test cases were generated test for failures in the program. Not
only were the test cases testing the functionality of the application, but we
also had to ensure that together, all lines of code in the program were
executed, and all branches were taken. Additionally, a control flow graph was
generated. For the second part of this lab, a conceptual exercise was done to
think about and discuss the benefits of pairwise testing versus exhaustive
testing, and the effectiveness of the tests generated using this method. We
also compared the tests generated by the PICT tool versus tests made using a
standard orthogonal array. 


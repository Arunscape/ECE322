The purpose of this lab was to serve as an introduction to cause-effect graphs
and decision tables. The goal was to become familiar with the use of these
graphs and rules in black box testing, and to put into practice generating test
cases using these tools. We tested two GUI applications written in Java.
The first application, InsuranceCalculator takes in 3 inputs, a gender
(Male/Female), an age, and number of claims. A cause-effect graph (Appendix A)
was derived from 5 rules which were given. From the cause-effect graph, a
decision table is derived, and test cases were made using the decision table.
Depending on the input, this application outputs either \$750, \$1000, \$1500,
or \$3000. The second appplication, BoilerShutdown is also a Java GUI
application. It takes in 10 inputs labelled from A-J, which are checkboxes.
Depending on the inputs, it will output either OK or FAIL. In this case, a
cause-effect graph was provided, however, it was unsimplified. By tracing back,
a simplified cause-effect graph was created (Appendix B), from which a decision
table was derived. Finally, from the decision table, test cases were created.
The purpose of cause-effect graphs is to divide the software specification into
workable pieces. Once a decision table is generated, the test cases are derived
using boundary value analysis, and also looking at the columns of the decision
table. 

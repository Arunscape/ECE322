In this lab, we were introduced to the white-box testing strategy. There was a
focus on control flow testing, and coverage criterion: Statement coverage,
branch coverage, condition coverage, and path coverage. A control flow graph
was generated, and test cases were made knowing the implementation of the
program, and by inspecting to source code. Knowing the implementation details
of the program allowed for the creation of test cases that not only check for
correctness of the implementation of the algorithm, but also that all lines and
branches in the code were covered by the tests. Statement and branch coverage on
their own do not indicate good tests, but they can at a glance tell us which
parts of an application for sure have not been tested yet. In this part of the
lab, we also discovered the infeasability of path coverage, due to a while loop
causing the number of potential paths taken to increase exponentially. In the
second part of
the lab, the focus was on pairwise testing, where we analyzed a hypothetical
testing scenario and used a standard orthogonal array to generate test cases,
and contrasted that with using a tool made my Microsoft named PICT which
automatically generates test cases, and compared and contrasted these methods
versus exhaustive testing. In general, it seems like pairwise testing is great
for reducing the number of test cases, while still keeping detection rate of
failures high, but it also has some weaknesses, like when more than two inputs
have some important relation in the application, when pairwise testing by
definition only looks at the relationships between each pair of inputs.

In this lab, we were introduced to integration testing.  We tested a simple
command-line database application using two integration testing strategies:
non-incrementeal testing, and the bottum up incremental testing strategy.

In general, integration testing seems beneficial over testing individual
modules. It allows us to verify that the system as a whole is working as
expected. For example, we noticed that as a whole, functions like updateData
worked as expected, even though they failed when testing the modules
individually. 

This type of testing might be tedious for a large scale system, but would
definitely be effective, since it would ensure that each module is communicating
with other modules as expected. If the modules were tested individually for a
large scale system without integration, we might find that the system as a whole
does not work, even though the unit tests are passing. Furthermore, using
inegration testing strategies allows for the errors to be found and traced to
the modules where the error is happening fairly easily. Imagine trying to debug
a large scale system as a black box\dots



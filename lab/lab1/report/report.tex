%preamble
\documentclass[letterpaper]{article}
\synctex=1
\usepackage{graphicx}
\graphicspath{ {images/} }

\usepackage{lipsum}
\usepackage{float}

\usepackage{amssymb}

\usepackage{siunitx}

\usepackage{multirow}
% for merging table cells I think

\usepackage{tabularx}
% allows for linewrap within cells
\newcolumntype{Y}{>{\centering\arraybackslash}X}

\usepackage{todonotes}
\usepackage{hyperref}


\title{ECE 322 \\
Lab Reporrt 1}
\author{Arun Woosaree\\
XXXXXXX}
\begin{document}
 \maketitle 

 \section*{Introduction}
 The purpose of this lab was to serve as a practical introduction to rudimentary black-box testing techniques.
 The testing methods introduced were dirty testing, error guessing, and partition-based testing.
 It should be noted that numerous other black-box testing methods exist
 The idea of black-box testing is that tests are carried out with no knowledge of how the software
 internally works. In other words, the implementation details are a ``black box'' as the name would suggest.
 


\section*{Part 1}

\section*{Part 2}

\appendix
\url{https://docs.google.com/spreadsheets/d/1ypM8AWWF2TXz8ygNpfAepDfbU3O3hC7fkOhFA33VGng/edit#gid=584397887}
\todo{delet this}
\end{document}

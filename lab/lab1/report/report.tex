%preamble
\documentclass[letterpaper]{article}
\synctex=1
\usepackage{graphicx}
\graphicspath{ {images/} }

\usepackage{lipsum}
\usepackage{float}

\usepackage{amssymb}

\usepackage{siunitx}

\usepackage{multirow}
% for merging table cells I think

\usepackage{tabularx}
\renewcommand\tabularxcolumn[1]{m{#1}}% for vertical centering text in X column

% allows for linewrap within cells
\newcolumntype{Y}{>{\centering\arraybackslash}X}

\usepackage{todonotes}
\usepackage{hyperref}

\usepackage{pdfpages} % for attaching the table lol

% \usepackage[toc,page]{appendix}


\title{ECE 322 \\
Lab Report 1}
\author{Arun Woosaree\\
XXXXXXX}
\begin{document}
\maketitle

\section*{Introduction}
The purpose of this lab was to serve as a practical introduction to rudimentary black-box testing techniques.
The testing methods introduced were dirty testing, error guessing, and partition-based testing.
It should be noted that numerous other black-box testing methods exist
The idea of black-box testing is that tests are carried out with no knowledge of how the software
internally works. In other words, the implementation details are a ``black box'' as the name would suggest.
Dirty testing and error guessing involves using creativity to come up with test cases,
and also using past experiences to come up with test cases to find faults in the program.
THe purpose of partition-based testing is to categorize possible test cases in `equivalence'
classes, and to test as many valid equivalence classes with as few test cases, and to come up
with a test case for each invalid equivalence class. The goal for partition-based testing
is to lower the number of test cases.

\section*{Part 1 - Failure/Dirty Testing, Error Guessing}
For task one in this lab, we had to be creative, as is the nature of Failure/Dirty testing, and error guessing.
The purpose was to test the functionality of a calculator program, which was written in Java. A table of
test cases was produced, checking for basic functionality, common errors. A few test cases were also
made based on previous experience, which is also known as error guessing. Altogether, the test cases check
for the following functionality:

\begin{enumerate}
    \item whether the calculator buttons work
    \item non-numerical input
    \item mismatched brackets
    \item order or operations (BEDMAS/PEMDAS)
    \item large numbers
    \item small numbers
    \item incorrect syntax (e.g. 2++2)
\end{enumerate}

The full list of test cases, along with the inputs and expected versus actual outputs can be found in Appendix \ref{calculatortestcases}.
The test cases where the expected result does not match the actual result are highlighted in red.

\todo{explain the failed test cases}


\section*{Part 2 - Partition Testing}
Task two of this lab involved partition-based testing of a triangle application.
The purpose of this application is to take 3 space separated positive integers,
each representing sides of a triangle, and the program is expected to tell the
user whether the triangle is a scalene, isosceles, or equilateral triangle.
The following equivalence classes were decided on for creating the test cases.
\subsection*{Triangle Equivalence Classes}

% \subsubsection*{Valid}
% \begin{enumerate}
%     \item $a + b > c$
%     \item Equilateral
%     \item Isosceles
%     \item Scalene
%     \item 3 arguments
%     \item separated by one space
%     \item positive integers
%     \item Enter command pressed after input
% \end{enumerate}

% \subsubsection*{Invalid}
% \begin{enumerate}
%     \setcounter{enumi}{7}
%     \item $a + b = c$
%     \item $ a + b < c$
%     \item $< 3$ arguments
%     \item $> 3$ arguments
%     \item separated by more than one space
%     \item negative argument
%     \item argument with the number `0'
%     \item decimal argument
%     \item Enter command not presses after input
% \end{enumerate}

\begin{table}[H]
    \begin{tabularx}{1.1\textwidth}{X|X|X|}
        \centering
        Input Condition                       & Valid Input Classes   & Invalid Input Classes                         \\ \hline
        number of input arguments             & 3 input arguments (1) & $< 3$ input arguments (9)                     \\
                                              &                       & $> 3$ input arguments (10)                    \\ \hline
        space between arguments               & one space (2)         & more than one space (11)                      \\
                                              &                       & non space character separating arguments (12) \\ \hline
        argument type                         & positive integer (3)  & negative integer (13)                         \\
                                              &                       & zero (14)                                     \\
                                              &                       & decimal (15)                                  \\ \hline
        triangle type                         & equilateral (4)       & $a + b = c$ (16)                              \\
                                              & isosceles (5)         & $a + b < c$ (17)                              \\
                                              & scalene (6)           &                                               \\
                                              & $a + b > c$ (7)       &                                               \\ \hline
        key pressed after inputting arguments & Enter key pressed (8) & Enter key not pressed (18)                    \\ \hline
    \end{tabularx}
    \caption{Valid and Invalid equivalence classes for the triangle program}
    \label{equivalenceclasses}
\end{table}

From these equivalence classes, the following test cases were created:

\subsection*{Test cases for valid inputs}
\begin{itemize}
    \item 3 3 3 covers (1, 2, 3, 4, 7, 8)
    \item 4 4 5 covers (1, 2, 3, 5, 7, 8)
    \item 6 7 8 covers (1, 2, 3, 6, 7, 8)
\end{itemize}

\subsection*{Test cases for invalid inputs}
\begin{itemize}
    \item 1 2 covers (9)
    \item 3 4 5 6 covers (10)
    \item 7 8       9 covers (11)
    \item 8\_7\_6 covers (12)
    \item 1 $-2$ 3 covers (13)
    \item 5 0 4 covers (14)
    \item 3 2 0.1 covers (15)
    \item 2 2 4 covers (16)
    \item 3 4 9 covers (17)
    \item not pressing Enter covers (18)
\end{itemize}


\appendix
\section*{Appendix}
\section{Calculator Test Cases} \label{calculatortestcases}
\includepdf[pages=-]{calculatortable.pdf}

% \begin{appendices}
% \section{Calculator Test Cases}
% \includepdf[pages=-]{calculatortable.pdf}
% \end{appendices}
\end{document}

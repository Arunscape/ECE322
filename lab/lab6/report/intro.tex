The purpose of this lab was to serve as an introduction to mutation testing
techniques, as well as regression testing.

Mutation testing uses the idea of auomatic code mutation to determine how well
the tests a tester writes actually are at catching errors in the application.
The idea is the change, or `mutate' the source code by changing equality
operator, mathematical operaors, among other things. Each copy of mutated code
is referred to as a `mutant', and the same set of tests is run on the mutants.
In order to do the mutation testing, we must first begin with a test suite where
all the tests are green. The idea is that when a mutant fails a test, it is
`killed' and the goal is to make sure that your test suite kills all mutants.
Any mutants that survive are a sign that the tests could be improved, since they
did not catch the bugs the mutants introduced. In this lab, we test an Array
Library program using the mutation testing technique. The program is written in
Java, and was tested using JUnit 5, and the PIT testing library was used to
generate the mutants and evaluate whether they survive the test suite or not.

With regression testing, we are concerned with testing a system again after an
update is made to the code base. For example, when a new feature or `bug fix' is
made, the regression test suite is run to make sure that the change to the code
base does not re-introduce bugs into the system. The idea is that for each
iteration of the software, when a  bug is found, a bugfix is applied, and a test
is also written which checks for that bug. That way, when a new feature is
applied or some other change is made to the codebase, we can make sure that the
new changes do not regress and bring back bugs which were previously already
fixed. In this lab, we test a Math library program using the regression testing
technique. The program is written in Java and a test suite was created with
JUnit 5. 

The project was built using Java 13.
A \textbf{build.gradle} is provided for ease of use,
from which an IDE like Intellij or Eclipse should be able to install
dependencies from and run the tests for both projects. Alternatively, the
command
\texttt{./gradlew test} can be run from the command-line.
\texttt{./gradlew pitest} can be run from the command-line for part 1 of this
lab to generate the mutation test report. 
The test suites for each project are located in \texttt{src/main/test/java}
relative to the project root. 

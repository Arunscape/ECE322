The purpose of this lab was to serve as an introduction to integration testing.
Integration testing is a logical extension of the previous lab, in which we did
unit testing. This also allows us to build on the JUnit skills we learned in
the previous lab. We were also introduced to the idea of mocking, using the
Mockito library to mock out relationships between classes as we desire.
Generally, there are two strategies to integration testing. The first is
\textit{Non-incremental} testing, also known as Big Bang testing. With this
strategy, each module is tested individually, and then there is one final test
where the entire system is tested as a whole. The second integration testing
approach is \textit{Incremental} testing, in which we combine the next module
to be tested with the set of previously tested modules before running tests.
This can be done in either a bottom up or top down approach. In this lab, a
simple command-line database program written in Java was tested. There are 7
modules which make up the system.  Module A is responsible for the GUI, B for
opening, C for sorting, D for modifying, E for exiting, F for displaying, and G
for updating data. Files in the database contain one entry per line, which are
comma separated. Both testing strategies were used to test the application in
this lab. Big bang testing was done in the first part, and we chose the
Bottum-up incremental testing strategy for the second portion of the lab. Where
possible, the tests were crafted to cover the full functionality of the
program, including statement coverage. The exceptions are modules A and B,
where it was not feasible to cover some lines.  The case with Module A was that
line 147 cannot be covered because when the program exits, an exception is
thrown, which halts execution. With Module B, testing lines 39-42 when an IO
Exception is caught does not need to be tested, because there is already code
which handles the case when a file is not found.  Furthermore, this situation
was never encountered in our testing, due to the error handing when a file is
not found. Additionally, the only code that gets run if this IO Exception
somehow gets thrown are library functions (Printing to stdout, and printing a
stacktrace of an error), and these library functions are likely already tested
extensively, otherwise programmers would not trust them.  The tests were
written using \textbf{junit-jupiter:5.5.2}, \textbf{mockito-core:3.1.0}, and
were run using Java 13 with the command-line argument
\textbf{--enable-preview}. A \textbf{build.gradle} is provided for ease of use,
from which an IDE like Intellij or Eclipse should be able to install
dependencies from and run the tests.

For part 2 of this lab, a second Java application was tested using another
testing technique known as regression testing. This application is a Math
library which implements some common functions such as max, generating random
arrays, standard deviation, and more. A summary of all the test cases discussed
can be found in Appendix \ref{part2table}.

A first iteration of the program was tested, in which two bugs were found,
described below. There were a total of nine test cases for this part.

\begin{enumerate}
  \item \underline{normalizeTest}:
          The normalize method almost works, but there is a small bug where it
          subtracts the $i_{th}$ element from the minimum, when it should
          actually be $values[i] - min$
          In \texttt{Commit.java} this fix was implemented.
  \item \underline{arrayAddTestDifferentLength}:
          This test fails because the method assumes that both arrays given to
          it are the same length. The test expects an AssertionFailed error.
          This test case was not fixed in \texttt{Commit.java}
\end{enumerate}

Next, there was an incremental upgrade to the code in the Commit.java file.  The
same test suite was run on it, and also new test cases were created to cover new
functionality of the application. These test cases are outlined in Appendix
\ref{part2table}. In this iteration, four new test cases were added for the new
functionality. Six out of thirteen test cases failed.  Using a diff tool, we see
that some boundary conditions were changed in the max and min methods, but they
ultimately did not affect the correctness of the program, since the difference
is that it now sets the max or min to the element in the array if it is greater
than or equal to the maximum instead of strictly greater than, and for minimum,
it sets the minimum if the element is less than or equal to the minimum instead
of strictly less than. Either way, the global maximum and minimum will be the
same. 

\begin{enumerate}
  \item \underline{arrayDeviationTest}:
    This method returns null, no matter the input. It should return an array.
  \item \underline{arrayAddTestDifferentLength}:
    Again, this does not assert that the input arrays have the name length
    This test fails for the same reason as the previous test did, so nothing
    changed here.
  \item \underline{distanceTest}:
    This method returns zero, regardless of the input
  \item \underline{arraySubtractTestDifferentLength}:
    Again, this does not assert that the input arrays have the name length
  \item \underline{negateTest}:
    This method now returns a value which is one less than the expected output.
    This test was passing before, but it now fails due to a new change, which
    subtracts one from each array element in addition to negating it.
  \item \underline{arraySubractTestSameLength}:
    Again, this does not assert that the input arrays have the name length
\end{enumerate}

While most of the new functionality failed tests anyways, due to not being fully
implemented yet, one regression was caught using the regression testing method.
At first glance, the negation method seems to have been inexplicably modified to
also subtract one from each element. Because the array subtract method depends
on negate, this test also fails. 


Finally, for this part of the lab, we applied fixes to the software. The new
code is available in Appendix \ref{part2code}. With the fixes all thirteen test
cases now pass. The full summary of test cases and their results can be found in
Appendix \ref{part2table}. Most of the fixes involve using existing
functionality that the Java standard library provides, instead of re-inventing
the wheel. These library functions have been battle tested by millions of users
and tests, and are much less likely to have bugs than anything we write
independently.

Regression testing, as the name implies is excellent for making sure one does
not regress when adding new features to software. That is, when a bug is found, a
test is made, so that when new functionality is added to the program, we ensure
the bug does not show up again, and if it is re-introduced, the test will catch
it. The lab demonstrated how well this technique works with the test for the
negation method catching a bug which was introduced in the new commit where
someone made it also subtract one from each element in the array in addition to
negating the element.

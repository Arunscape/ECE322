In this lab, we were introduced to mutation and regression testing.  We tested
two applications written in Java. The first application is an Array library.
This was tested with the mutation testing strategy, using a library called
PITest. 
In general, mutation testing seems to be a useful tool for getting a feel for
how robust a testing suite is. By generating mutant code, and making sure the
mutants are caught and killed by the testing suite, we gain confidence in the
robustness of the testing suite for each mutant killed. Like any other testing
strategy, it is not a silver bullet, but I can see this technique being used as
a `litmus test' of sorts to give an initial impression for how good a test suite
is. After all, if your test suite does not catch the mutants, and the code base
is later updated when a new feature is added for example and the feature
introduces a bug, the system likely will not work as expected. With a more robust
test suite, it is more likely that changes like this will be caught by the
tests, and therefore won't be introduced to the system. Because computing
mutants takes up a considerable amount of resources, I can see this technique
working well for small to medium-sized systems. I think that for larger systems,
computing the mutants and running the test suites may be too time-consuming.

We also tested a Math application for this lab using the regression testing
strategy. In general, regression testing seems to be really useful for what it
is meant for: making sure previously found bugs are not re-introduced into the
system. I can see it being useful for small and large systems alike. By making
sure known bugs are not reintroduced into the system, developer time and effort
is not wasted dealing with issues more than once, since the tests that cover
each regression test should do the work of catching the bugs as opposed to the
developer trying to trace an issue which was previously fixed before, but some
new feature undid the previous fix. 

Overall, both mutation testing and regression testing techniques seem to be
really useful and practical testing techniques. Mutation testing is more
concerned with the quality of the test suite, while regression testing is
concerned with not re-introducing old bugs, and to prevent regression as the name
suggests. I think that these techniques should not be used alone, but rather
together with other testing techniques learned over the course of the labs.
Making sure your test suite is robust enough to catch small changes is
important, as is the ability to automatically detect and prevent a regression
when a new feature is added to an application. Used in combination with other
testing techniques, these are very powerful. 

%preamble
\documentclass[letterpaper]{article}
\synctex=1
\usepackage{graphicx}
\graphicspath{ {images/} }

\usepackage{lipsum}
\usepackage{float}
% \bibliographystyle{IEEEtran}
% \bibliographystyle{ieeetr}

\usepackage{amssymb}

\usepackage{siunitx}

\usepackage{multirow}
% for merging table cells I think

\usepackage{tabularx}
% allows for linewrap within cells
\newcolumntype{Y}{>{\centering\arraybackslash}X}

\usepackage{todonotes}


\usepackage{fancyhdr} %header
\fancyhf{}
\fancyhead[R]{Arun Woosaree XXXXXXX}
\renewcommand\headrulewidth{0pt}
\fancyfoot[C]{\thepage}
\renewcommand\footrulewidth{0pt}
\pagestyle{fancy}

% make subsection use letters
\renewcommand{\thesubsection}{\thesection\ \alph{subsection})}


% \usepackage{amsthm}

%actual document
\begin{document}

% \maketitle %insert titlepage here
\begin{titlepage}
 \begin{center}
  \vspace*{1cm}
  \Huge
  ECE 322
  \vspace{1cm}
  
  Assignment 1
  \vspace{1cm}
  
  Arun Woosaree
  
  \today
  \vfill
 \end{center}
\end{titlepage}

\section{}
After reading the two papers, the two most essential factors which make software testing difficult in my opinion is: 

\begin{enumerate}
	\item human nature
	\item the always changing nature of software
\end{enumerate}

\section{}

\begin{itemize}
	\item Functionality
		Does the software allow the vehicle to work autonomously, without human input? 
		Does it do so in an acceptable manner? 
		(i.e. safely, and reaching the destination in about the same amount of time as a good human driver would do, or better.
	\item Performance and reliability
		Does the software still control the vehicle in an acceptable manner in more difficult situations?
		If certain conditions like heavier traffic puts a higher load on the processing unit for the system, does the softare still behave reliably?
		It should work on both highways and in city driving conditions. 
		It should still control the vehicle reliable when conditions are non-ideal.
		(i.e. it would be unnaceptable for the system to lose control on slippery roads)
	\item Efficiency
	\item Maintainability
	\item Usability
	\item Portability
\end{itemize}

\begin{table}[H]
\centering
\begin{tabular}{c|c|c|}
Risk Category & Technical Risk & Business Risk \\ \hline
Risk 1 &  &  \\ \hline
Risk 2 &  &  \\ \hline
Risk 3 &  &  \\ \hline
\end{tabular}
\end{table}

\end{document}

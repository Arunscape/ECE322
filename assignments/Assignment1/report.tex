%preamble
\documentclass[letterpaper]{article}
\synctex=1
\usepackage{graphicx}
\graphicspath{ {images/} }

\usepackage{lipsum}
\usepackage{float}

\usepackage[
    style=ieee,
    backend=biber
    ]{biblatex}
\addbibresource{references.bib}

\usepackage{hyperref}

\usepackage{amssymb}

\usepackage{siunitx}

\usepackage{multirow}
% for merging table cells I think

\usepackage{tabularx}
% allows for linewrap within cells
\newcolumntype{Y}{>{\centering\arraybackslash}X}

\usepackage{todonotes}


\usepackage{fancyhdr} %header
\fancyhf{}
\fancyhead[R]{Arun Woosaree XXXXXXX}
\renewcommand\headrulewidth{0pt}
\fancyfoot[C]{\thepage}
\renewcommand\footrulewidth{0pt}
\pagestyle{fancy}

% make subsection use letters
\renewcommand{\thesubsection}{\thesection\ \alph{subsection})}


% \usepackage{amsthm}

%actual document
\begin{document}

% \maketitle %insert titlepage here
\begin{titlepage}
 \begin{center}
  \vspace*{1cm}
  \Huge
  ECE 322
  \vspace{1cm}
  
  Assignment 1
  \vspace{1cm}
  
  Arun Woosaree
  
  \today
  \vfill
 \end{center}
\end{titlepage}

\section{}
After reading the two papers, the two most essential factors which make software testing difficult in my opinion is: 

\begin{enumerate}
	\item Human nature\\
		In my opinion, people are generally lazy. In terms of software testing, this means that in some cases,
		people will tend to write enough tests to convince themselves that their program is correct, or 
		In some cases they might even refuse to write tests once they have some code that in their mind already works, 
		so there is no percieved need to test the software.
		This is probably because most programmers experience a significant mental reward when they finally get a program `working'.
		However, when it comes to writing tests for the code they just wrote, there usually is much less excitement.
		Sometimes, programmers may even find themselves wrestling with a testing framework, which can be a frustrating experience when (they think that)
		their program is already working as expected.

	\item the always changing nature of software

\end{enumerate}

\section{}

\begin{itemize}
	\item Functionality
		
	\begin{itemize}
		\item Does the software allow the vehicle to work autonomously, without human input? 
		\item Does it do so in an acceptable manner? 
		\item (i.e. does it reach the destination in about the same amount of time as a good human driver would do or better?\\
		 Does it do it as safe as, or better than a good human driver?)
	\end{itemize}
	\item Performance and reliability
	\begin{itemize}
		\item How reliably does the software react to its environment?
		\item Does the software still control the vehicle in an acceptable manner in more difficult situations?
		\item If certain conditions like heavier traffic puts a higher load on the processing unit for the system, does the softare still behave reliably?
		\item It should work in different driving conditions (e.g. highways vs in-city, sunny vs slippery roads)
		\item Will the software perform just as reliably a few years from now?
	\end{itemize}
		\item Efficiency
	\begin{itemize}
		\item Does the software utilize its resources efficiently?
		\item Does the software respond quickly enough to its environment?
	\end{itemize}
	\item Maintainability
	\begin{itemize}	
	\item Is the software well-documented?
		\item How easy is it to add functionality to the software? (e.g. if new driving laws have to be followed, how easy will it be to add a patch to be in compliance)
		\item How much technical debt exists in the software project
		\item When faults
		% \todo{check if this is the right word} 
		are found, how easy is it to fix them?
	\end{itemize}
	\item Usability
	\begin{itemize}	
	\item From the user's perspective, how easy is it to use the autonomous mode of the vehicle, and how is the user experience?
		\item Also, how easy it is to switch between autonomous and manual modes of the vehicle?
	\end{itemize}
	\item Portability
	\begin{itemize}	
	\item Can the software be used in multiple types of vehicles? 
		\item (e.g. if there are multiple models of cars, can the same software be used with all of them?)
	\end{itemize}
\end{itemize}

\begin{table}[H]
\centering
\begin{tabular}{c|c|c|}
Risk Category & Technical Risk & Business Risk \\ \hline
Car might crash while in autonomous mode &  & Might get sued (1)  \\ \hline
Poor quality tests & (1)  &  \\ \hline
Poor test coverage & (2) &  \\ \hline
Poorly defined requirements & (1) & \\ \hline
\end{tabular}
\end{table}

\subsection{delet this}
technical risk is the likelihood that a fault might exist

assessment: system architect, designer, senior programmers


business risk is the impact of a given ault might have on the users,
customers, and other stakeholders

1 - very high 5- very low risk

\section{}

\begin{enumerate}
	\item Reliability (Operation factor)
	\item Interoperability (Transition factor)
	\item Usability (Operation factor)
	\item Efficiency (Operation factor)
\end{enumerate}

\section{}


	\subsection*{Feature Description}

	\subsubsection*{Boeing 737 MAX flight control system}

	\subsection*{Nature of Software Failure}

	\subsection*{Any testing efforts regarding the failure?}
	apparently it was ``only'' classified as a “major failure,” meaning that it could cause physical distress to people on the plane, but not death. \cite{gates_2019}

	\subsection*{Any follow up action taken? Any plan to alleviate further problems?}

	\section{todo, read these articles}
	\begin{itemize}
		\item https://spectrum.ieee.org/aerospace/aviation/how-the-boeing-737-max-disaster-looks-to-a-software-developer
		\item https://www.bloomberg.com/news/articles/2019-06-28/boeing-s-737-max-software-outsourced-to-9-an-hour-engineers
		\item https://www.bloomberg.com/news/articles/2019-07-27/latest-737-max-fault-that-alarmed-test-pilots-rooted-in-software
		\item https://www.google.com/search?q=boeing+737+max+software+testing&oq=boeing+737+max+software+testing&aqs=chrome..69i57.6818j0j0&sourceid=chrome&ie=UTF-8
		\item https://www.businessinsider.com/boeing-knew-737-max-software-error-year-before-telling-faa-2019-5
		\item https://www.cnn.com/2019/06/26/politics/boeing-737-max-flaw/index.html
		\item https://www.theverge.com/2019/5/2/18518176/boeing-737-max-crash-problems-human-error-mcas-faa
	\end{itemize}
\nocite{*}
\printbibliography
\end{document}
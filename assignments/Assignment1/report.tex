%preamble
\documentclass[letterpaper]{article}
\synctex=1
\usepackage{graphicx}
\graphicspath{ {images/} }

\usepackage{lipsum}
\usepackage{float}
% \bibliographystyle{IEEEtran}
% \bibliographystyle{ieeetr}

\usepackage{amssymb}

\usepackage{siunitx}

\usepackage{multirow}
% for merging table cells I think

\usepackage{tabularx}
% allows for linewrap within cells
\newcolumntype{Y}{>{\centering\arraybackslash}X}

\usepackage{todonotes}


\usepackage{fancyhdr} %header
\fancyhf{}
\fancyhead[R]{Arun Woosaree XXXXXXX}
\renewcommand\headrulewidth{0pt}
\fancyfoot[C]{\thepage}
\renewcommand\footrulewidth{0pt}
\pagestyle{fancy}

% make subsection use letters
\renewcommand{\thesubsection}{\thesection\ \alph{subsection})}


% \usepackage{amsthm}

%actual document
\begin{document}

% \maketitle %insert titlepage here
\begin{titlepage}
 \begin{center}
  \vspace*{1cm}
  \Huge
  ECE 322
  \vspace{1cm}
  
  Assignment 1
  \vspace{1cm}
  
  Arun Woosaree
  
  \today
  \vfill
 \end{center}
\end{titlepage}

\section{}
After reading the two papers, the two most essential factors which make software testing difficult in my opinion is: 

\begin{enumerate}
	\item human nature
	\item the always changing nature of software
\end{enumerate}

\section{}

\begin{itemize}
	\item Functionality
		Does the software allow the vehicle to work autonomously, without human input? 
		Does it do so in an acceptable manner? 
		(i.e. does it reach the destination in about the same amount of time as a good human driver would do or better?
		does it do it as safe as, or better than a good human driver?)
	\item Performance and reliability
		How reliably does the software react to its environment?
		Does the software still control the vehicle in an acceptable manner in more difficult situations?
		If certain conditions like heavier traffic puts a higher load on the processing unit for the system, does the softare still behave reliably?
		It should work in different driving conditions (e.g. highways vs in-city, sunny vs slippery roads)
		Will the software perform just as reliably a few years from now?
	\item Efficiency
		Does the software utilize its resources efficiently?
		Does the software respond quickly enough to its environment?
	\item Maintainability
		Is the software well-documented?
		How easy is it to add functionality to the software? (e.g. if new driving laws have to be followed, how easy will it be to add a patch to be in compliance)
		How much technical debt exists in the software project
		When faults\todo{check if this is the right word} are found, how easy is it to fix them?
	\item Usability
		From the user's perspective, how easy is it to use the autonomous mode of the vehicle, and how is the user experience?
		Also, how easy it is to switch between autonomous and manual modes of the vehicle?
	\item Portability
		Can the software be used in multiple types of vehicles? 
		(e.g. if there are multiple models of cars, can the same software be used with all of them?)
\end{itemize}

\begin{table}[H]
\centering
\begin{tabular}{c|c|c|}
Risk Category & Technical Risk & Business Risk \\ \hline
Risk 1 &  &  \\ \hline
Risk 2 &  &  \\ \hline
Risk 3 &  &  \\ \hline
\end{tabular}
\end{table}

\end{document}
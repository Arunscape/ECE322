\documentclass[letterpaper]{article}
\synctex=1
\usepackage{graphicx}
\graphicspath{ {images/} }

\usepackage{lipsum}
\usepackage{float}

% \usepackage[
%     style=ieee,
%     backend=biber
%     ]{biblatex}
% \addbibresource{references.bib}

\usepackage{hyperref}

\usepackage{amssymb}

\usepackage{siunitx}

\usepackage{multirow}
% for merging table cells I think

\usepackage{tabularx}
\renewcommand\tabularxcolumn[1]{m{#1}}% for vertical centering text in X column
% allows for linewrap within cells
\newcolumntype{Y}{>{\centering\arraybackslash}X}

\usepackage{todonotes}
\usepackage{pdfpages}

\usepackage{fancyhdr} %header
\fancyhf{}
\fancyhead[R]{Arun Woosaree XXXXXXX}
\renewcommand\headrulewidth{0pt}
\fancyfoot[C]{\thepage}
\renewcommand\footrulewidth{0pt}
\pagestyle{fancy}

\usepackage[pdf]{graphviz}
\usepackage{adjustbox}

\usepackage{amsmath}

% make subsection use letters
\renewcommand{\thesubsection}{\alph{subsection})}

\usepackage{listings}
\lstdefinestyle{mystyle}{
%    backgroundcolor=\color{backcolour},
%    commentstyle=\color{codegreen},
%    keywordstyle=\color{magenta},
    numberstyle=\tiny\color{codegray},
%    stringstyle=\color{codepurple},
%    basicstyle=\ttfamily\footnotesize,
%    breakatwhitespace=false,
%    breaklines=true,
%    captionpos=b,
%    keepspaces=true,
%    numbers=left,
%    numbersep=5pt,
%    showspaces=false,
%    showstringspaces=false,
%    showtabs=false,
%    tabsize=2
}


% \usepackage{amsthm}
\title{ECE 322 \\
Assignment 2}
  \author{Arun Woosaree\\
  XXXXXXX}
%actual document
\begin{document}

\maketitle %insert titlepage here

\section{}
To test the individual states, we are concerned with the stationary
probabilies. The following linear system of equations is derived from the
graph: (Note: there was a typo addressed by the professor. The transition from
C to B has a weight of 1. In general, the sum of probabilities of edges leaving
any node is equal to 1.)

  \[ p_A = 0.5p_E \]
  \[ p_B = p_A + p_C \]
  \[ p_C = 0.5p_E + 0.6B \]
  \[ p_D = 0.4p_B \]
  \[ p_E = p_D \]

The last equation is dropped, because otherwise the system would be linearly
dependent. $p_D$ is determined from the following relationship:

  \[ p_A+p_B+p_C+p_D+p_E = 1 \]

The system of equations is solved using the following code:

\vspace{1cm}
\begin{adjustbox}{width=\textwidth}
\lstinputlisting[language=Python]{stationaryprob.py}
\end{adjustbox}
\vspace{2cm}

which results in:
\\
$p_B$: 0.3571428571428571\\
$p_C$: 0.28571428571428564\\
$p_E$: 0.1428571428571429\\
$p_D$: 0.14285714285714288\\
$p_A$: 0.07142857142857145\\

Thus, the order in which to test the individual states is: B, C, E, D, A.
\vspace{1cm}

As for testing the transitions, we look at the probabilities of the
transitions, and pick the highest ones first. Therefore, the transitions I
would test first are:
\\
$A\to B$\\
$D\to E$\\
$C\to B$\\

, since they all have the highest probability of 1

\section{}
Unknown relationships: $4 \times 5 \times 2 \times 5 \times 2 \times 3 = 1200$
test cases.

Dependencies known: 
$4 \times 5 \times 2 = 40$ for x, 
$5$ for y, and
$2 \times 3 = 6$ for z.
Summing these together, we get 51 total test cases as an upper bound if the
relationships are known.



\end{document}

\documentclass[letterpaper]{article}
\synctex=1
\usepackage{graphicx}
\graphicspath{ {images/} }

\usepackage{lipsum}
\usepackage{float}

% \usepackage[
%     style=ieee,
%     backend=biber
%     ]{biblatex}
% \addbibresource{references.bib}

\usepackage{hyperref}

\usepackage{amssymb}

\usepackage{siunitx}

\usepackage{multirow}
% for merging table cells I think

\usepackage{tabularx}
\renewcommand\tabularxcolumn[1]{m{#1}}% for vertical centering text in X column
% allows for linewrap within cells
\newcolumntype{Y}{>{\centering\arraybackslash}X}

\usepackage{todonotes}


\usepackage{fancyhdr} %header
\fancyhf{}
\fancyhead[R]{Arun Woosaree XXXXXXX}
\renewcommand\headrulewidth{0pt}
\fancyfoot[C]{\thepage}
\renewcommand\footrulewidth{0pt}
\pagestyle{fancy}

\usepackage[pdf]{graphviz}
\usepackage{adjustbox}

\usepackage{amsmath}
 
% make subsection use letters
\renewcommand{\thesubsection}{\alph{subsection})}


% \usepackage{amsthm}
\title{ECE 322 \\
Assignment 2}
  \author{Arun Woosaree\\
  XXXXXXX} 
%actual document
\begin{document}

\maketitle %insert titlepage here

\section{Credit Union}
\begin{adjustbox}{center}
	\begin{tabularx}{\textwidth}{llll}
Conditions                                & what & the & fuck \\ \cline{1-1}
male AND city dweller                     &      &     &      \\
male AND age $<25$                         &      &     &      \\
female AND $25<$ age $<65$ AND NOT city dweller &      &     &      \\
NOT (male AND age $>65$)                   &      &     &      \\
					  &      &     &      \\
Actions                                   &      &     &      \\ \cline{1-1}
Show Product A                            &      &     &      \\
Show Product B                            &      &     &      \\
Show Product C                            &      &     &      \\
Show Product D                            &      &     &     
\end{tabularx}%
\end{adjustbox}
\subsection{Maximal number of rules}

\subsection{}

\section{}
For the given subdomain, the following lines form the boundaries:
 \begin{itemize}                                                                 
         \item $y=5, 0\leq x\leq 7$                                              
         \item $x=0, 0\leq y \leq 5$                                             
         \item $y=-x, 0\leq x\leq 1$                                             
         \item $y=x-2, 1\leq x \leq 7$                                           
 \end{itemize}

\subsection{EPC Strategy}
From the boundary lines, we see that the maximum value that $x$ can have is 
$7$, its minumum is $-1$, and that the maximum value that $y$ can have is
5 while its minumum value is 0. Using the EPC testing strategy, 
$4^2 + 1=17$ test cases are expected. The extreme points chosen are
$(7, 7.1, 0, -0.1)$ for $x$, and $(5, 5.1, 0, -0.1)$ for $y$.
For the additional test case within the boundary, $(x=1, y=1)$ is chosen.
The full list of suggested test cases is found below:


\begin{adjustbox}{center}
	\begin{tabular}{lll}
		test id & x    & y    \\ \hline
1       & 7    & 5    \\
2       & 7    & 5.1  \\
3       & 7    & -1   \\
4       & 7    & -0.1 \\
5       & 7.1  & 5    \\
6       & 7.1  & 5.1  \\
7       & 7.1  & -1   \\
8       & 7.1  & -0.1 \\
9       & 0    & 5    \\
10      & 0    & 5.1  \\
11      & 0    & -1   \\
12      & 0    & -0.1 \\
13      & -0.1 & 5    \\
14      & -0.1 & 5.1  \\
15      & -0.1 & -1   \\
16      & -0.1 & -0.1 \\
17      & 1    & 1    \\
\end{tabular}
\end{adjustbox}

\subsection{Weak n x 1 Strategy}
Given that there are 4 boundaries, we expect $4(2+1) +1 = 13$ test cases. 
The dimensionality is 2, so 2 points are chosen on each boundary, as well
as one additional point just outside of each boundary. The last test case
is one point inside the boundaries. The full list of suggested test cases
is found below: 
\begin{adjustbox}{center}
	\begin{tabular}{llll}
		test id & description 			        & x & y    \\ \hline
		1       & on $y=5, 0\leq x\leq 7$ boundary      & 2 & 5    \\
		2       & on $y=5, 0\leq x\leq 7$ boundary      & 4 & 5    \\
		3       & outside $y=5, 0\leq x\leq 7$ boundary & 3 & 5.1 \\
		4 & on $x=0, 0\leq y \leq 5$ boundary           & 0 & 2 \\
		5 & on $x=0, 0\leq y \leq 5$ boundary           & 0 & 4 \\
		6 & outside $x=0, 0\leq y \leq 5$ boundary      & -0.1 & 3 \\
		7 & on $y=-x, 0\leq x\leq 1$ boundary           & 0.3 & -0.3 \\
		8 & on $y=-x, 0\leq x\leq 1$ boundary           & 0.7 & -0.7 \\
		9 & outside  $y=-x, 0\leq x\leq 1$ boundary      & 0.5 & -0.6 \\
		10 & on $y=x-2, 1\leq x \leq 7$ boundary        & 3 & 1 \\
		11 & on $y=x-2, 1\leq x \leq 7$ boundary        & 5 & 3 \\
		12 & outside $y=x-2, 1\leq x \leq 7$ boundary   & 4 & 1.9 \\
		13 & Inside the boundaries                      & 1 & 1 \\


                                                      
                                                      
                                                    
\end{tabular}
\end{adjustbox}
\section{Cause-Effect Graph}

\section{Combinatorial Testing}
\end{document}

\documentclass[letterpaper]{article}
\synctex=1
\usepackage{graphicx}
\graphicspath{ {images/} }

\usepackage{lipsum}
\usepackage{float}

\usepackage[
    style=ieee,
    backend=biber
    ]{biblatex}
\addbibresource{references.bib}

\usepackage{hyperref}

\usepackage{amssymb}

\usepackage{siunitx}

\usepackage{multirow}
% for merging table cells I think

\usepackage{tabularx}
% allows for linewrap within cells
\newcolumntype{Y}{>{\centering\arraybackslash}X}

\usepackage{todonotes}


\usepackage{fancyhdr} %header
\fancyhf{}
\fancyhead[R]{Arun Woosaree XXXXXXX}
\renewcommand\headrulewidth{0pt}
\fancyfoot[C]{\thepage}
\renewcommand\footrulewidth{0pt}
\pagestyle{fancy}

\usepackage[pdf]{graphviz}
\usepackage{adjustbox}

\usepackage{amsmath}
% make subsection use letters
\renewcommand{\thesubsection}{\thesection\ \alph{subsection})}


% \usepackage{amsthm}

%actual document
\begin{document}

% \maketitle %insert titlepage here
\begin{titlepage}
	\begin{center}
		\vspace*{1cm}
		\Huge
		ECE 322
		\vspace{1cm}

		Assignment 1
		\vspace{1cm}

		Arun Woosaree

		\today
		\vfill
	\end{center}
\end{titlepage}

\section{e-Shopping System FSM}

The following assumptions were made:
\begin{enumerate}
	\item The items added to the user's online shopping cart are always
		in stock
	\item A user must sign up before being able to purchase an item
		from this e-shopping system.
	\item Once the user has signed up, their account cannot be deleted.
		They can however, remain logged out indefinitely.
	\item Once an order is processed, the user cannot cancel their order
\end{enumerate}
\begin{adjustbox}{width=2\textwidth,center}
\digraph{eshopping}{
	rankdir=LR;
	size="8,5"
	// node [shape = doublecircle];
	node [shape = circle];
	begin [label= "", shape=none,height=.0,width=.0];
	begin -> "Sign up" [label = "User signs up"];
	"Sign up" -> "Shopping" [label = "User logs in"];
	"Logged out" -> Shopping [ label = "User logs in" ];
	Shopping -> Shopping [label = "Add item to cart"];
	Shopping -> Checkout [label = "User clicks checkout"];
	Shopping -> "Logged out" [label = "User logs out"];
	Checkout -> Shopping [label = "User cancels order"];
	Checkout -> "Logged out" [label = "User Logs out"];
	Checkout -> "Process order" [label = "Checkout successful"];
	"Process order" -> "Shopping";


}
\end{adjustbox}

\section{maxofThreeNumbers(int n1, int n2, int n3}
\subsection{Exhaustive Testing}
By definition, with exhaustive testing, we would have to check for every
possible combination of inputs to cover the input space. 
Assuming the program in question stores its
\textbf{int} data type as a 64-bit signed integer, each parameter can
have a minimum value of $-9223372036854775808$, and a maximum value of
$9223372036854775807$. Therefore, for each input argument, there are
$18446744073709551615$ possibilities. So, to account for each possible
combination of inputs, there would be 
\begin{multline*}
18446744073709551615 \times 18446744073709551615 \times 18446744073709551615 = \\ 6277101735386680762814942322444851025767571854389858533375
\end{multline*}
test cases. 

\subsection{Error Guessing}
With error guessing, we can choose some inputs from the input space
that from previous experience and from guessing we might think could 
break the program. A few test cases are listed below:

\begin{enumerate}
	\item maxOfThreeNumbers(-1, 0 2) checks for negative and positive inputs
	\item maxOfThreeNumbers(0, 0, 1) checks for when two inputs are the same
	\item maxOfThreeNumbers(-9223372036854775808, 0 4) minmum value for one input
	\item maxOfThreeNumbers(2, -2, 9223372036854775807) maximum value for one input
	\item maxOfThreeNumbers(0, 0, 0) checks for when all arguments are zero, and also when all the arguments are the same
	\item maxOfThreeNumbers(1, 2, 3) checks for all positive arguments
	\item maxOfThreeNumbers(-5, -9, -2) checks for all negative arguments
\end{enumerate}

\section{}

\nocite{*}
\printbibliography
\end{document}

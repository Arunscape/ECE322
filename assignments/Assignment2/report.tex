\documentclass[letterpaper]{article}
\synctex=1
\usepackage{graphicx}
\graphicspath{ {images/} }

\usepackage{lipsum}
\usepackage{float}

% \usepackage[
%     style=ieee,
%     backend=biber
%     ]{biblatex}
% \addbibresource{references.bib}

\usepackage{hyperref}

\usepackage{amssymb}

\usepackage{siunitx}

\usepackage{multirow}
% for merging table cells I think

\usepackage{tabularx}
\renewcommand\tabularxcolumn[1]{m{#1}}% for vertical centering text in X column
% allows for linewrap within cells
\newcolumntype{Y}{>{\centering\arraybackslash}X}

\usepackage{todonotes}


\usepackage{fancyhdr} %header
\fancyhf{}
\fancyhead[R]{Arun Woosaree XXXXXXX}
\renewcommand\headrulewidth{0pt}
\fancyfoot[C]{\thepage}
\renewcommand\footrulewidth{0pt}
\pagestyle{fancy}

\usepackage[pdf]{graphviz}
\usepackage{adjustbox}

\usepackage{amsmath}

\usepackage{listings}
\usepackage{xcolor}
 
\definecolor{codegreen}{rgb}{0,0.6,0}
\definecolor{codegray}{rgb}{0.5,0.5,0.5}
\definecolor{codepurple}{rgb}{0.58,0,0.82}
\definecolor{backcolour}{rgb}{0.95,0.95,0.92}
 
\lstdefinestyle{mystyle}{
%    backgroundcolor=\color{backcolour},   
%    commentstyle=\color{codegreen},
%    keywordstyle=\color{magenta},
    numberstyle=\tiny\color{codegray},
%    stringstyle=\color{codepurple},
%    basicstyle=\ttfamily\footnotesize,
%    breakatwhitespace=false,         
%    breaklines=true,                 
%    captionpos=b,                    
%    keepspaces=true,                 
%    numbers=left,                    
%    numbersep=5pt,                  
%    showspaces=false,                
%    showstringspaces=false,
%    showtabs=false,                  
%    tabsize=2
}
 
\lstset{style=mystyle}

% make subsection use letters
\renewcommand{\thesubsection}{\thesection\ \roman{subsection})}


% \usepackage{amsthm}
\title{ECE 322 \\
Assignment 2}
  \author{Arun Woosaree\\
  XXXXXXX} 
%actual document
\begin{document}

\maketitle %insert titlepage here
%\begin{titlepage}
%	\begin{center}
%		\vspace*{1cm}
%		\Huge
%		ECE 322
%		\vspace{1cm}
%
%		Assignment 1
%		\vspace{1cm}
%
%		Arun Woosaree
%
%		\today
%		\vfill
%	\end{center}
%\end{titlepage}

\section{e-Shopping System FSM}

The following assumptions were made:
\begin{enumerate}
	\item The items added to the user's online shopping cart are always
	      in stock
	\item A user must sign up before being able to purchase an item
	      from this e-shopping system.
	\item Once the user has signed up, their account cannot be deleted.
	      They can however, remain logged out indefinitely.
	\item Once an order is processed, the user cannot cancel their order
\end{enumerate}
\begin{adjustbox}{width=2\textwidth,center}
	\digraph{eshopping}{
	rankdir=LR;
	size="8,5";
	overlap=false;
	// node [shape = doublecircle];
	node [shape = circle];
	begin [label= "", shape=none,height=.0,width=.0];
	begin -> "Sign up" [label = "User signs up"];
	"Sign up" -> "Shopping" [label = "User logs in"];
	"Logged out" -> Shopping [ label = "User logs in" ];
	Shopping:n -> Shopping:n [label = "Add item to cart"];
	Shopping:s -> Shopping:s [label = "Remove item from cart"];
	Shopping -> Checkout [label = "User clicks checkout"];
	Shopping -> "Logged out" [label = "User logs out"];
	Checkout -> Shopping [label = "User cancels order"];
	Checkout -> "Logged out" [label = "User Logs out"];
	Checkout -> "Process order" [label = "Checkout successful"];
	"Process order" -> "Shopping";


	}
\end{adjustbox}

\section{maxofThreeNumbers(int n1, int n2, int n3}
\subsection{Exhaustive Testing}
By definition, with exhaustive testing, we would have to check for every
possible combination of inputs to cover the input space.
Assuming the program in question stores its
\textbf{int} data type as a 64-bit signed integer, each parameter can
have a minimum value of $-9223372036854775808$, and a maximum value of
$9223372036854775807$. Therefore, for each input argument, there are
$18446744073709551615$ possibilities. So, to account for each possible
combination of inputs, there would be
\begin{multline*}
	18446744073709551615 \times 18446744073709551615 \times 18446744073709551615 = \\ 6277101735386680762814942322444851025767571854389858533375
\end{multline*}
test cases.

\subsection{Error Guessing}
With error guessing, we can choose some inputs from the input space
that from previous experience and from guessing we might think could
break the program. A few test cases are listed below:

\begin{enumerate}
	\item maxOfThreeNumbers(-1, 0 2) checks for negative and positive inputs
	\item maxOfThreeNumbers(0, 0, 1) checks for when two inputs are the same
	\item maxOfThreeNumbers(-9223372036854775808, 0 4) minimum value for one input
	\item maxOfThreeNumbers(2, -2, 9223372036854775807) maximum value for one input
	\item maxOfThreeNumbers(0, 0, 0) checks for when all arguments are zero, and also when all the arguments are the same
	\item maxOfThreeNumbers(1, 2, 3) checks for all positive arguments
	\item maxOfThreeNumbers(-5, -9, -2) checks for all negative arguments
	\item maxofThreeNumbers( 'a', 'b', 'c' ) checks for non integer arguments
\end{enumerate}

\section{Equivalence Partitioning}
\subsection{equivalence classes}
Given $n$ input variables and $m$ equivalence classes in each $n^{th}$
input space, there would be $m \times n$ total equivalence classes.
There would be at most one test case for each 
%test case with a valid input for
equivalence class using strong normal equivalence class testing.
Using weak normal equivalence class testing, the number of test
cases can be lowered by covering as many valid input equivalence classes
with as few test cases as possible.
%that equivalence class (because multiple valid equivalence classes can be
%covered with one test case), and at least one test case for each invalid input
%for that equivalence class. 
Thus, we would have at most $m \times n $
test cases. As we've been implying, this is an upper bound, and the number of
test cases can absolutely be reduced using the single fault assumption. 
We still need one test case for each invalid input equivalence class, but one
test case can cover multiple valid inputs for different equivalence classes
using weak normal equivalence class testing.
%Thus, the minimum number
%of test cases would be $m \times n + 1$, which would be the case when
%all valid equivalence classes can be covered by one test case.

For example, for $n=10$ and $m=10$, the most number of test cases can be
calculated as follows:
\[ 10 \times 10 = 100 \]
%And, if there is an input that covers all valid equivalence classes, the
%minimum number of test cases would be
%\[ 10 \times 10 + 1 = 101 \] test cases.

\subsection{example with function S}
given:
\begin{enumerate}
	\item input range [-50, 50]
	\item S is invoked if the reading of a sensor is within [a.b] or [c,d], $b < c$
\end{enumerate}

\begin{adjustbox}{center}
	% \begin{table}[H]
	\begin{tabularx}{1.2\textwidth}{X|X|X|}
		%        \centering
		Input Condition & Valid Input Classes      & Invalid Input Classes    \\ \hline
		sensor reading  & reading within [a,b] (1) & reading $<a$ (3)         \\
				& reading within [c,d] (2) & reading $>d$ (4)         \\
				&                          & $b <$ reading $< c$ (5)  \\ \hline
	\end{tabularx}
	% \caption{Valid and Invalid equivalence classes for the triangle program}
	% \end{table}
\end{adjustbox}

\subsubsection*{Valid input test cases}
\begin{itemize}
	\item an input between a and b would cover (1). e.g. if a was -25 and b
		was 10, 0 would suffice
	\item and input between c and d would cover (2). e.g if c was 20 and d
		was 30, then 20 would suffice
\end{itemize}

\subsubsection*{Invalid input test cases}
\begin{itemize}
	\item an input less than a would cover (3). e.g. if a was -25 then -30
		would suffice
	\item an input greater than d would cover (4). e.g. if d was 30 then 40
		would suffice
	\item an input between b and c would cover (5). e.g. if b was 10 and c
		was 20 then 15 would suffice.
\end{itemize}

\subsection{Generalizing the problem in (ii)}
With two sensors, function S can be invoked if either of the sensors are within
the range $[a_i, b_i]$ or $[c_i, d_i]$, $i=1,2$. 
It does not seem like we can cover multiple valid input equivalence classes
using weak normal equivalence class testing, so using strong normal equivalence 
class testing, we would take the Cartesian product of the equivalence classes. 
In (ii) we had 5 equivalence classes, so the Cartesian product is 
$5 \times 5 = 25$. So, there would be 25 equivalence classes for the valid
inputs, since we could have any combination of either one or both sensors being
within their ranges, and we would again have one test case for each equivalence
class resulting in 25 total test cases.
%As for the equivalence classes for invalid inputs, there
%are 9 cases where both sensors should not invoke the function. In this case, we
%cannot cover more than one equivalence class with a singular test case, so in
%total there would be 17
%total test cases.

\section{3-D Input Domain}
For $W_1$, $W_2$, and $W_3$ to form a partition, they must be mutually
exclusive. Using a naive Python script, I determined with up to 0.0001 accuracy
that $e$ must be between $0 < e < 5$. If there was no restriction that e
must be positive, then it seems that any negative number would suffice as well.

%\begin{lstlisting}[language=Python]
%import numpy as np                                                                  
%X = [i for i in range(0, 11)]                                                       
%Y = [i for i in range(-5, 21)]                                                      
%Z = [i for i in range(0, 8)]                                                        
%                                                                                     
%possible_values_of_e = np.arange(-30, 101, 0.0001).tolist()                         
%w = set([ (x, y, z) for x in X for y in Y for z in Z])                              
%                                                                                      
%                                                                                      
%for e in possible_values_of_e:                                                      
%    w1 = set([ i for i in w if max(abs(i[0]-1), abs(i[1]-1), abs(i[2]-1)) <= e])    
%    w2 = set([ i for i in w if max(abs(i[0]-5), abs(i[1]-10), abs(i[2]-4)) <= e])
%    w3 = w - w1 - w2                                                                                                                                                  
%    
%    # w3 is automatically disjoint from w1 and w2 bevause w3 = w - w1 - w2  
%    if len( w1 & w2  ) == 0:                                                        
%        print(e)
%\end{lstlisting}
\vspace{1cm}
\begin{adjustbox}{width=\textwidth}
\lstinputlisting[language=Python]{q4.py}
\end{adjustbox}

\vspace{2cm}
Thus, it would appear that $e$ can take on any numerical value between
0 and 5 exclusive at both ends. 
\[ 0 < e < 5 \]
% \nocite{*}
% \printbibliography
\end{document}

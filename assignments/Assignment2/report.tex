\documentclass[letterpaper]{article}
\synctex=1
\usepackage{graphicx}
\graphicspath{ {images/} }

\usepackage{lipsum}
\usepackage{float}

% \usepackage[
%     style=ieee,
%     backend=biber
%     ]{biblatex}
% \addbibresource{references.bib}

\usepackage{hyperref}

\usepackage{amssymb}

\usepackage{siunitx}

\usepackage{multirow}
% for merging table cells I think

\usepackage{tabularx}
\renewcommand\tabularxcolumn[1]{m{#1}}% for vertical centering text in X column
% allows for linewrap within cells
\newcolumntype{Y}{>{\centering\arraybackslash}X}

\usepackage{todonotes}


\usepackage{fancyhdr} %header
\fancyhf{}
\fancyhead[R]{Arun Woosaree XXXXXXX}
\renewcommand\headrulewidth{0pt}
\fancyfoot[C]{\thepage}
\renewcommand\footrulewidth{0pt}
\pagestyle{fancy}

\usepackage[pdf]{graphviz}
\usepackage{adjustbox}

\usepackage{amsmath}
% make subsection use letters
\renewcommand{\thesubsection}{\thesection\ \alph{subsection})}


% \usepackage{amsthm}

%actual document
\begin{document}

% \maketitle %insert titlepage here
\begin{titlepage}
	\begin{center}
		\vspace*{1cm}
		\Huge
		ECE 322
		\vspace{1cm}

		Assignment 1
		\vspace{1cm}

		Arun Woosaree

		\today
		\vfill
	\end{center}
\end{titlepage}

\section{e-Shopping System FSM}

The following assumptions were made:
\begin{enumerate}
	\item The items added to the user's online shopping cart are always
	      in stock
	\item A user must sign up before being able to purchase an item
	      from this e-shopping system.
	\item Once the user has signed up, their account cannot be deleted.
	      They can however, remain logged out indefinitely.
	\item Once an order is processed, the user cannot cancel their order
\end{enumerate}
\begin{adjustbox}{width=2\textwidth,center}
	\digraph{eshopping}{
	rankdir=LR;
	size="8,5"
	// node [shape = doublecircle];
	node [shape = circle];
	begin [label= "", shape=none,height=.0,width=.0];
	begin -> "Sign up" [label = "User signs up"];
	"Sign up" -> "Shopping" [label = "User logs in"];
	"Logged out" -> Shopping [ label = "User logs in" ];
	Shopping -> Shopping [label = "Add item to cart"];
	Shopping -> Checkout [label = "User clicks checkout"];
	Shopping -> "Logged out" [label = "User logs out"];
	Checkout -> Shopping [label = "User cancels order"];
	Checkout -> "Logged out" [label = "User Logs out"];
	Checkout -> "Process order" [label = "Checkout successful"];
	"Process order" -> "Shopping";


	}
\end{adjustbox}

\section{maxofThreeNumbers(int n1, int n2, int n3}
\subsection{Exhaustive Testing}
By definition, with exhaustive testing, we would have to check for every
possible combination of inputs to cover the input space.
Assuming the program in question stores its
\textbf{int} data type as a 64-bit signed integer, each parameter can
have a minimum value of $-9223372036854775808$, and a maximum value of
$9223372036854775807$. Therefore, for each input argument, there are
$18446744073709551615$ possibilities. So, to account for each possible
combination of inputs, there would be
\begin{multline*}
	18446744073709551615 \times 18446744073709551615 \times 18446744073709551615 = \\ 6277101735386680762814942322444851025767571854389858533375
\end{multline*}
test cases.

\subsection{Error Guessing}
With error guessing, we can choose some inputs from the input space
that from previous experience and from guessing we might think could
break the program. A few test cases are listed below:

\begin{enumerate}
	\item maxOfThreeNumbers(-1, 0 2) checks for negative and positive inputs
	\item maxOfThreeNumbers(0, 0, 1) checks for when two inputs are the same
	\item maxOfThreeNumbers(-9223372036854775808, 0 4) minimum value for one input
	\item maxOfThreeNumbers(2, -2, 9223372036854775807) maximum value for one input
	\item maxOfThreeNumbers(0, 0, 0) checks for when all arguments are zero, and also when all the arguments are the same
	\item maxOfThreeNumbers(1, 2, 3) checks for all positive arguments
	\item maxOfThreeNumbers(-5, -9, -2) checks for all negative arguments
\end{enumerate}

\section{Equivalence Partitioning}
\subsection{equivalence classes}
Given $n$ input variables and $m$ equivalence classes in each $n^{th}$
input space, there would be $m \times n$ total equivalence classes.
There would be at most one test case for each test case with a valid input for
that equivalence class (because multiple valid equivalence classes can be
covered with one test case), and at least one test case for each invalid input
for that equivalence class. Thus, we would have at most $m \times n \times 2$
test cases. As we've been implying, this is an upper bound, and the number of
test cases can absolutely be reduced. We still need one test case for each
equivalence class's invalid input, but one test case can cover multiple
valid inputs for different equivalence classes. Thus, the minimum number
of test cases would be $m \times n + 1$, which would be the case when
all valid equivalence classes can be covered by one test case.

For example, for $n=10$ and $m=10$, the most number of test cases can be
calculated as follows:
\[ 10 \times 10 \times 2 = 200 \]
And, if there is an input that covers all valid equivalence classes, the
minimum number of test cases would be
\[ 10 \times 10 + 1 = 101 \] test cases.

\subsection{example with function S}
given:
\begin{enumerate}
	\item input range [-50, 50]
	\item S is invoked if the reading of a sensor is within [a.b] or [c,d], $b < c$
\end{enumerate}

\begin{adjustbox}{center}
	% \begin{table}[H]
	\begin{tabularx}{1.2\textwidth}{X|X|X|}
		%        \centering
		Input Condition & Valid Input Classes      & Invalid Input Classes    \\ \hline
		sensor reading  & reading within [a,b] (1) & reading $<a$ (3)         \\
				& reading within [c,d] (2) & reading $>d$ (4)         \\
				&                          & $b <$ reading $< c$ (5)  \\ \hline
	\end{tabularx}
	% \caption{Valid and Invalid equivalence classes for the triangle program}
	% \end{table}
\end{adjustbox}

\subsubsection*{Valid input test cases}
\begin{itemize}
	\item an input between a and b would cover (1). e.g. if a was -25 and b
		was 10, 0 would suffice
	\item and input between c and d would cover (2). e.g if c was 20 and d
		was 30, then 20 would suffice
\end{itemize}

\subsubsection*{Invalid input test cases}
\begin{itemize}
	\item an input less than a would cover (3). e.g. if a was -25 then -30
		would suffice
	\item an input greater than d would cover (4). e.g. if d was 30 then 40
		would suffice
	\item an input between b and c would cover (5). e.g. if b was 10 and c
		was 20 then 15 would suffice.
\end{itemize}
% \nocite{*}
% \printbibliography
\end{document}
